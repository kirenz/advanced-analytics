% Options for packages loaded elsewhere
\PassOptionsToPackage{unicode}{hyperref}
\PassOptionsToPackage{hyphens}{url}
%
\documentclass[
]{book}
\usepackage{lmodern}
\usepackage{amssymb,amsmath}
\usepackage{ifxetex,ifluatex}
\ifnum 0\ifxetex 1\fi\ifluatex 1\fi=0 % if pdftex
  \usepackage[T1]{fontenc}
  \usepackage[utf8]{inputenc}
  \usepackage{textcomp} % provide euro and other symbols
\else % if luatex or xetex
  \usepackage{unicode-math}
  \defaultfontfeatures{Scale=MatchLowercase}
  \defaultfontfeatures[\rmfamily]{Ligatures=TeX,Scale=1}
\fi
% Use upquote if available, for straight quotes in verbatim environments
\IfFileExists{upquote.sty}{\usepackage{upquote}}{}
\IfFileExists{microtype.sty}{% use microtype if available
  \usepackage[]{microtype}
  \UseMicrotypeSet[protrusion]{basicmath} % disable protrusion for tt fonts
}{}
\makeatletter
\@ifundefined{KOMAClassName}{% if non-KOMA class
  \IfFileExists{parskip.sty}{%
    \usepackage{parskip}
  }{% else
    \setlength{\parindent}{0pt}
    \setlength{\parskip}{6pt plus 2pt minus 1pt}}
}{% if KOMA class
  \KOMAoptions{parskip=half}}
\makeatother
\usepackage{xcolor}
\IfFileExists{xurl.sty}{\usepackage{xurl}}{} % add URL line breaks if available
\IfFileExists{bookmark.sty}{\usepackage{bookmark}}{\usepackage{hyperref}}
\hypersetup{
  pdftitle={Advanced Analytics with Tidymodels},
  pdfauthor={Prof.~Dr.~Jan Kirenz},
  hidelinks,
  pdfcreator={LaTeX via pandoc}}
\urlstyle{same} % disable monospaced font for URLs
\usepackage{color}
\usepackage{fancyvrb}
\newcommand{\VerbBar}{|}
\newcommand{\VERB}{\Verb[commandchars=\\\{\}]}
\DefineVerbatimEnvironment{Highlighting}{Verbatim}{commandchars=\\\{\}}
% Add ',fontsize=\small' for more characters per line
\usepackage{framed}
\definecolor{shadecolor}{RGB}{248,248,248}
\newenvironment{Shaded}{\begin{snugshade}}{\end{snugshade}}
\newcommand{\AlertTok}[1]{\textcolor[rgb]{0.94,0.16,0.16}{#1}}
\newcommand{\AnnotationTok}[1]{\textcolor[rgb]{0.56,0.35,0.01}{\textbf{\textit{#1}}}}
\newcommand{\AttributeTok}[1]{\textcolor[rgb]{0.77,0.63,0.00}{#1}}
\newcommand{\BaseNTok}[1]{\textcolor[rgb]{0.00,0.00,0.81}{#1}}
\newcommand{\BuiltInTok}[1]{#1}
\newcommand{\CharTok}[1]{\textcolor[rgb]{0.31,0.60,0.02}{#1}}
\newcommand{\CommentTok}[1]{\textcolor[rgb]{0.56,0.35,0.01}{\textit{#1}}}
\newcommand{\CommentVarTok}[1]{\textcolor[rgb]{0.56,0.35,0.01}{\textbf{\textit{#1}}}}
\newcommand{\ConstantTok}[1]{\textcolor[rgb]{0.00,0.00,0.00}{#1}}
\newcommand{\ControlFlowTok}[1]{\textcolor[rgb]{0.13,0.29,0.53}{\textbf{#1}}}
\newcommand{\DataTypeTok}[1]{\textcolor[rgb]{0.13,0.29,0.53}{#1}}
\newcommand{\DecValTok}[1]{\textcolor[rgb]{0.00,0.00,0.81}{#1}}
\newcommand{\DocumentationTok}[1]{\textcolor[rgb]{0.56,0.35,0.01}{\textbf{\textit{#1}}}}
\newcommand{\ErrorTok}[1]{\textcolor[rgb]{0.64,0.00,0.00}{\textbf{#1}}}
\newcommand{\ExtensionTok}[1]{#1}
\newcommand{\FloatTok}[1]{\textcolor[rgb]{0.00,0.00,0.81}{#1}}
\newcommand{\FunctionTok}[1]{\textcolor[rgb]{0.00,0.00,0.00}{#1}}
\newcommand{\ImportTok}[1]{#1}
\newcommand{\InformationTok}[1]{\textcolor[rgb]{0.56,0.35,0.01}{\textbf{\textit{#1}}}}
\newcommand{\KeywordTok}[1]{\textcolor[rgb]{0.13,0.29,0.53}{\textbf{#1}}}
\newcommand{\NormalTok}[1]{#1}
\newcommand{\OperatorTok}[1]{\textcolor[rgb]{0.81,0.36,0.00}{\textbf{#1}}}
\newcommand{\OtherTok}[1]{\textcolor[rgb]{0.56,0.35,0.01}{#1}}
\newcommand{\PreprocessorTok}[1]{\textcolor[rgb]{0.56,0.35,0.01}{\textit{#1}}}
\newcommand{\RegionMarkerTok}[1]{#1}
\newcommand{\SpecialCharTok}[1]{\textcolor[rgb]{0.00,0.00,0.00}{#1}}
\newcommand{\SpecialStringTok}[1]{\textcolor[rgb]{0.31,0.60,0.02}{#1}}
\newcommand{\StringTok}[1]{\textcolor[rgb]{0.31,0.60,0.02}{#1}}
\newcommand{\VariableTok}[1]{\textcolor[rgb]{0.00,0.00,0.00}{#1}}
\newcommand{\VerbatimStringTok}[1]{\textcolor[rgb]{0.31,0.60,0.02}{#1}}
\newcommand{\WarningTok}[1]{\textcolor[rgb]{0.56,0.35,0.01}{\textbf{\textit{#1}}}}
\usepackage{longtable,booktabs}
% Correct order of tables after \paragraph or \subparagraph
\usepackage{etoolbox}
\makeatletter
\patchcmd\longtable{\par}{\if@noskipsec\mbox{}\fi\par}{}{}
\makeatother
% Allow footnotes in longtable head/foot
\IfFileExists{footnotehyper.sty}{\usepackage{footnotehyper}}{\usepackage{footnote}}
\makesavenoteenv{longtable}
\usepackage{graphicx}
\makeatletter
\def\maxwidth{\ifdim\Gin@nat@width>\linewidth\linewidth\else\Gin@nat@width\fi}
\def\maxheight{\ifdim\Gin@nat@height>\textheight\textheight\else\Gin@nat@height\fi}
\makeatother
% Scale images if necessary, so that they will not overflow the page
% margins by default, and it is still possible to overwrite the defaults
% using explicit options in \includegraphics[width, height, ...]{}
\setkeys{Gin}{width=\maxwidth,height=\maxheight,keepaspectratio}
% Set default figure placement to htbp
\makeatletter
\def\fps@figure{htbp}
\makeatother
\setlength{\emergencystretch}{3em} % prevent overfull lines
\providecommand{\tightlist}{%
  \setlength{\itemsep}{0pt}\setlength{\parskip}{0pt}}
\setcounter{secnumdepth}{5}
\usepackage{booktabs}
\usepackage[]{natbib}
\bibliographystyle{apalike}

\title{Advanced Analytics with Tidymodels}
\author{Prof.~Dr.~Jan Kirenz}
\date{2020-11-05}

\begin{document}
\maketitle

{
\setcounter{tocdepth}{1}
\tableofcontents
}
\hypertarget{welcome}{%
\chapter*{Welcome}\label{welcome}}
\addcontentsline{toc}{chapter}{Welcome}

This book provides an introduction to advanced analytics with R using the modeling packages called \href{https://tidymodels.org}{tidymodels} to build, evaluate, compare, and tune predictive models.

We'll cover key concepts in statistical learning and machine learning including overfitting, the holdout method, the bias-variance trade-off, ensembling, cross-validation, and feature engineering.

\hfill\break

\begin{center}\rule{0.5\linewidth}{0.5pt}\end{center}

This online book is licensed using the \href{https://creativecommons.org/licenses/by-nc/2.0/}{Creative Commons Attribution-NonCommercial 2.0 Generic (CC BY-NC 2.0) License}.

\hfill\break

\hypertarget{part-build-a-model}{%
\part{BUILD A MODEL}\label{part-build-a-model}}

\hypertarget{intro}{%
\chapter{Build a Model}\label{intro}}

\emph{This tutorial is based on Alisson Hill's excellent \href{https://alison.rbind.io/tags/tidymodels/}{tidymodels workshop}.}

In this tutorial you will learn how to specify a model with the tidymodels package.

As data, we use 2,930 houses sold in Ames, IA from 2006 to 2010, collected by the Ames Assessor's Office. From this dataset, we only select our dependent variable (\texttt{Sale\_Price}) and one predictor variable (\texttt{Gr\_Liv\_Area}).

\begin{Shaded}
\begin{Highlighting}[]
\KeywordTok{library}\NormalTok{(tidyverse)}
\end{Highlighting}
\end{Shaded}

\begin{verbatim}
## Warning: replacing previous import 'vctrs::data_frame' by 'tibble::data_frame'
## when loading 'dplyr'
\end{verbatim}

\begin{verbatim}
## -- Attaching packages --------------------------------------- tidyverse 1.3.0 --
\end{verbatim}

\begin{verbatim}
## v ggplot2 3.3.2     v purrr   0.3.4
## v tibble  3.0.4     v dplyr   1.0.0
## v tidyr   1.1.0     v stringr 1.4.0
## v readr   1.3.1     v forcats 0.5.0
\end{verbatim}

\begin{verbatim}
## -- Conflicts ------------------------------------------ tidyverse_conflicts() --
## x dplyr::filter() masks stats::filter()
## x dplyr::lag()    masks stats::lag()
\end{verbatim}

\begin{Shaded}
\begin{Highlighting}[]
\NormalTok{ames \textless{}{-}}\StringTok{ }\KeywordTok{read\_csv}\NormalTok{(}\StringTok{"https://raw.githubusercontent.com/kirenz/datasets/master/ames.csv"}\NormalTok{)}
\end{Highlighting}
\end{Shaded}

\begin{verbatim}
## Parsed with column specification:
## cols(
##   .default = col_character(),
##   Lot_Frontage = col_double(),
##   Lot_Area = col_double(),
##   Year_Built = col_double(),
##   Year_Remod_Add = col_double(),
##   Mas_Vnr_Area = col_double(),
##   BsmtFin_SF_1 = col_double(),
##   BsmtFin_SF_2 = col_double(),
##   Bsmt_Unf_SF = col_double(),
##   Total_Bsmt_SF = col_double(),
##   First_Flr_SF = col_double(),
##   Second_Flr_SF = col_double(),
##   Low_Qual_Fin_SF = col_double(),
##   Gr_Liv_Area = col_double(),
##   Bsmt_Full_Bath = col_double(),
##   Bsmt_Half_Bath = col_double(),
##   Full_Bath = col_double(),
##   Half_Bath = col_double(),
##   Bedroom_AbvGr = col_double(),
##   Kitchen_AbvGr = col_double(),
##   TotRms_AbvGrd = col_double()
##   # ... with 15 more columns
## )
\end{verbatim}

\begin{verbatim}
## See spec(...) for full column specifications.
\end{verbatim}

\begin{Shaded}
\begin{Highlighting}[]
\NormalTok{ames \textless{}{-}}\StringTok{ }
\StringTok{  }\NormalTok{ames }\OperatorTok{\%\textgreater{}\%}
\StringTok{  }\KeywordTok{select}\NormalTok{(Sale\_Price, Gr\_Liv\_Area)}

\KeywordTok{glimpse}\NormalTok{(ames)}
\end{Highlighting}
\end{Shaded}

\begin{verbatim}
## Rows: 2,930
## Columns: 2
## $ Sale_Price  <dbl> 215000, 105000, 172000, 244000, 189900, 195500, 213500,...
## $ Gr_Liv_Area <dbl> 1656, 896, 1329, 2110, 1629, 1604, 1338, 1280, 1616, 18...
\end{verbatim}

In this first example, we don't use data splitting.

\hypertarget{model-type}{%
\section{Model type}\label{model-type}}

\begin{enumerate}
\def\labelenumi{\arabic{enumi}.}
\tightlist
\item
  Pick an \texttt{model\ type}: choose from this \href{https://www.tidymodels.org/find/parsnip/}{list}
\item
  Set the \texttt{engine}: choose from this \href{https://www.tidymodels.org/find/parsnip/}{list}
\item
  Set the \texttt{mode}: regression or classification
\end{enumerate}

\begin{Shaded}
\begin{Highlighting}[]
\KeywordTok{library}\NormalTok{(tidymodels)}
\end{Highlighting}
\end{Shaded}

\begin{verbatim}
## -- Attaching packages -------------------------------------- tidymodels 0.1.0 --
\end{verbatim}

\begin{verbatim}
## v broom     0.5.6      v rsample   0.0.7 
## v dials     0.0.7      v tune      0.1.0 
## v infer     0.5.2      v workflows 0.1.1 
## v parsnip   0.1.2      v yardstick 0.0.6 
## v recipes   0.1.13
\end{verbatim}

\begin{verbatim}
## -- Conflicts ----------------------------------------- tidymodels_conflicts() --
## x scales::discard() masks purrr::discard()
## x dplyr::filter()   masks stats::filter()
## x recipes::fixed()  masks stringr::fixed()
## x dplyr::lag()      masks stats::lag()
## x yardstick::spec() masks readr::spec()
## x recipes::step()   masks stats::step()
\end{verbatim}

\begin{Shaded}
\begin{Highlighting}[]
\NormalTok{lm\_spec \textless{}{-}}\StringTok{ }\CommentTok{\# your model specification}
\StringTok{  }\KeywordTok{linear\_reg}\NormalTok{() }\OperatorTok{\%\textgreater{}\%}\StringTok{  }\CommentTok{\# model type}
\StringTok{  }\KeywordTok{set\_engine}\NormalTok{(}\DataTypeTok{engine =} \StringTok{"lm"}\NormalTok{) }\OperatorTok{\%\textgreater{}\%}\StringTok{  }\CommentTok{\# model engine}
\StringTok{  }\KeywordTok{set\_mode}\NormalTok{(}\StringTok{"regression"}\NormalTok{) }\CommentTok{\# model mode}

\CommentTok{\# Show your model specification}
\NormalTok{lm\_spec}
\end{Highlighting}
\end{Shaded}

\begin{verbatim}
## Linear Regression Model Specification (regression)
## 
## Computational engine: lm
\end{verbatim}

\hypertarget{model-training}{%
\section{Model training}\label{model-training}}

In the training process, you run an algorithm on data and thereby produce a model. This process is also called model fitting.

\begin{Shaded}
\begin{Highlighting}[]
\NormalTok{lm\_fit \textless{}{-}}\StringTok{ }\CommentTok{\# your fitted model}
\StringTok{  }\KeywordTok{fit}\NormalTok{( }
\NormalTok{  lm\_spec, }\CommentTok{\# your model specification }
\NormalTok{  Sale\_Price }\OperatorTok{\textasciitilde{}}\StringTok{ }\NormalTok{Gr\_Liv\_Area, }\CommentTok{\# a Linear Regression formula }
  \DataTypeTok{data =}\NormalTok{ ames }\CommentTok{\# your data}
\NormalTok{  )}

\CommentTok{\# Show your fitted model}
\NormalTok{lm\_fit}
\end{Highlighting}
\end{Shaded}

\begin{verbatim}
## parsnip model object
## 
## Fit time:  4ms 
## 
## Call:
## stats::lm(formula = Sale_Price ~ Gr_Liv_Area, data = data)
## 
## Coefficients:
## (Intercept)  Gr_Liv_Area  
##     13289.6        111.7
\end{verbatim}

\hypertarget{model-predictions}{%
\section{Model predictions}\label{model-predictions}}

We use our fitted model to make predictions.

\begin{Shaded}
\begin{Highlighting}[]
\NormalTok{price\_pred \textless{}{-}}\StringTok{ }
\StringTok{  }\NormalTok{lm\_fit }\OperatorTok{\%\textgreater{}\%}\StringTok{ }
\StringTok{  }\KeywordTok{predict}\NormalTok{(}\DataTypeTok{new\_data =}\NormalTok{ ames) }\OperatorTok{\%\textgreater{}\%}
\StringTok{  }\KeywordTok{mutate}\NormalTok{(}\DataTypeTok{price\_truth =}\NormalTok{ ames}\OperatorTok{$}\NormalTok{Sale\_Price)}

\KeywordTok{head}\NormalTok{(price\_pred)}
\end{Highlighting}
\end{Shaded}

\begin{verbatim}
## # A tibble: 6 x 2
##     .pred price_truth
##     <dbl>       <dbl>
## 1 198255.      215000
## 2 113367.      105000
## 3 161731.      172000
## 4 248964.      244000
## 5 195239.      189900
## 6 192447.      195500
\end{verbatim}

If we would want to make predictions for new houses, we could proceed as follows:

\begin{Shaded}
\begin{Highlighting}[]
\CommentTok{\# New values (our x variable)}
\NormalTok{new\_homes \textless{}{-}}\StringTok{ }
\StringTok{  }\KeywordTok{tibble}\NormalTok{(}\DataTypeTok{Gr\_Liv\_Area =} \KeywordTok{c}\NormalTok{(}\DecValTok{334}\NormalTok{, }\DecValTok{1126}\NormalTok{, }\DecValTok{1442}\NormalTok{, }\DecValTok{1500}\NormalTok{, }\DecValTok{1743}\NormalTok{, }\DecValTok{5642}\NormalTok{)) }

\CommentTok{\# Prediction for new houses (predict y)}
\NormalTok{lm\_fit }\OperatorTok{\%\textgreater{}\%}
\StringTok{ }\KeywordTok{predict}\NormalTok{(}\DataTypeTok{new\_data =}\NormalTok{ new\_homes)}
\end{Highlighting}
\end{Shaded}

\begin{verbatim}
## # A tibble: 6 x 1
##     .pred
##     <dbl>
## 1  50595.
## 2 139057.
## 3 174352.
## 4 180831.
## 5 207972.
## 6 643467.
\end{verbatim}

\hypertarget{model-evaluation}{%
\section{Model evaluation}\label{model-evaluation}}

We use the Root Mean Squared Error (RMSE) to evaluate our regression model. Therefore, we use the function \(rmse(data, truth, estimate)\).

\begin{Shaded}
\begin{Highlighting}[]
\KeywordTok{rmse}\NormalTok{(}\DataTypeTok{data =}\NormalTok{ price\_pred, }
     \DataTypeTok{truth =}\NormalTok{ price\_truth, }
     \DataTypeTok{estimate =}\NormalTok{ .pred)}
\end{Highlighting}
\end{Shaded}

\begin{verbatim}
## # A tibble: 1 x 3
##   .metric .estimator .estimate
##   <chr>   <chr>          <dbl>
## 1 rmse    standard      56505.
\end{verbatim}

\hypertarget{process-with-data-splitting}{%
\chapter{Process with data splitting}\label{process-with-data-splitting}}

The best way to measure a model's performance at predicting new data is to actually predict new data.

This function ``splits'' data randomly into a single testing and a single training set: \texttt{initial\_split(data,\ prop\ =\ 3/4,\ strata,\ breaks)}. We also use \href{https://en.wikipedia.org/wiki/Stratified_sampling}{stratified sampling} in this example.

\hypertarget{data-splitting}{%
\section{Data splitting}\label{data-splitting}}

\begin{Shaded}
\begin{Highlighting}[]
\KeywordTok{set.seed}\NormalTok{(}\DecValTok{100}\NormalTok{) }

\NormalTok{ames\_split \textless{}{-}}\StringTok{  }\KeywordTok{initial\_split}\NormalTok{(ames,}
                             \DataTypeTok{strata =}\NormalTok{ Sale\_Price,}
                             \DataTypeTok{breaks =} \DecValTok{4}\NormalTok{)}

\NormalTok{ames\_train \textless{}{-}}\StringTok{  }\KeywordTok{training}\NormalTok{(ames\_split)}
\NormalTok{ames\_test \textless{}{-}}\StringTok{ }\KeywordTok{testing}\NormalTok{(ames\_split)}
\end{Highlighting}
\end{Shaded}

\hypertarget{model-type-1}{%
\section{Model type}\label{model-type-1}}

\begin{Shaded}
\begin{Highlighting}[]
\NormalTok{lm\_spec\_}\DecValTok{2}\NormalTok{ \textless{}{-}}
\StringTok{  }\KeywordTok{linear\_reg}\NormalTok{() }\OperatorTok{\%\textgreater{}\%}\StringTok{  }
\StringTok{  }\KeywordTok{set\_engine}\NormalTok{(}\DataTypeTok{engine =} \StringTok{"lm"}\NormalTok{) }\OperatorTok{\%\textgreater{}\%}\StringTok{  }
\StringTok{  }\KeywordTok{set\_mode}\NormalTok{(}\StringTok{"regression"}\NormalTok{) }
\end{Highlighting}
\end{Shaded}

\hypertarget{model-training-1}{%
\section{Model training}\label{model-training-1}}

\begin{Shaded}
\begin{Highlighting}[]
\NormalTok{lm\_fit\_}\DecValTok{2}\NormalTok{ \textless{}{-}}\StringTok{ }
\StringTok{  }\NormalTok{lm\_spec\_}\DecValTok{2} \OperatorTok{\%\textgreater{}\%}\StringTok{ }
\StringTok{  }\KeywordTok{fit}\NormalTok{(Sale\_Price }\OperatorTok{\textasciitilde{}}\StringTok{ }\NormalTok{Gr\_Liv\_Area,}
      \DataTypeTok{data =}\NormalTok{ ames\_train) }\CommentTok{\# only use training data}
\end{Highlighting}
\end{Shaded}

\hypertarget{model-predictions-1}{%
\section{Model predictions}\label{model-predictions-1}}

\begin{Shaded}
\begin{Highlighting}[]
\NormalTok{price\_pred\_}\DecValTok{2}\NormalTok{ \textless{}{-}}\StringTok{ }
\StringTok{  }\NormalTok{lm\_fit }\OperatorTok{\%\textgreater{}\%}\StringTok{ }
\StringTok{  }\KeywordTok{predict}\NormalTok{(}\DataTypeTok{new\_data =}\NormalTok{ ames\_test) }\OperatorTok{\%\textgreater{}\%}\StringTok{ }
\StringTok{  }\KeywordTok{mutate}\NormalTok{(}\DataTypeTok{price\_truth =}\NormalTok{ ames\_test}\OperatorTok{$}\NormalTok{Sale\_Price)}
\end{Highlighting}
\end{Shaded}

\hypertarget{model-evaluation-1}{%
\section{Model evaluation}\label{model-evaluation-1}}

\begin{Shaded}
\begin{Highlighting}[]
\KeywordTok{rmse}\NormalTok{(price\_pred\_}\DecValTok{2}\NormalTok{, }
     \DataTypeTok{truth =}\NormalTok{ price\_truth, }
     \DataTypeTok{estimate =}\NormalTok{ .pred)}
\end{Highlighting}
\end{Shaded}

\begin{verbatim}
## # A tibble: 1 x 3
##   .metric .estimator .estimate
##   <chr>   <chr>          <dbl>
## 1 rmse    standard      59054.
\end{verbatim}

\hypertarget{import}{%
\chapter{Import}\label{import}}

\hypertarget{transformation}{%
\chapter{Transformation}\label{transformation}}

\hypertarget{part-basics}{%
\part{BASICS}\label{part-basics}}

\hypertarget{model-building}{%
\chapter{Build a model}\label{model-building}}

\hypertarget{build-models-with-resampling}{%
\chapter{Build models with resampling}\label{build-models-with-resampling}}

  \bibliography{book.bib,packages.bib}

\end{document}
